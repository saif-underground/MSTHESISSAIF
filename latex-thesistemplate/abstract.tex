% abstract.tex (Abstract)

\addcontentsline{toc}{chapter}{Abstract}

\chapter*{Abstract}

Accurate electricity demand forecasting is an important problem as the failure to do so may be costly for both economic and environmental reasons. Power TAC simulation system provides a non-risky platform to do research on smart grid based energy generation and distribution system. Brokers are important components of the system. The brokers work as self-interested entities that want to maximize their profits by buying energy in a lower cost and selling it to a higher cost. To be successful, a broker has to make forecast of the electricity demand about its customers as accurately as possible otherwise it will be unbale to make profit. The proposed forecasting method uses a combination of cluster and classifiers. At first, the customers are clustered based on their weekly average usage. After that, energy usage history and related weather related information of the customers in the same cluster are combined together to train classifier for the cluster. To forecast about an unseen customer, the proposed method needs at least a week's energy usage history of the customer. The system assigns the  unseen customer to one of the clusters based on its energy usage history. The classifier for that cluster will be used to make forecast about the customer. This approach produced 13 \% error compared to 31\% relative absolute error observed from the moving average baseline predictor. Again, Power TAC system has six different types of customer such as customers with demand shifting capability, customer with no demand shifting capability, electric vehicle, thermal storage, wind production and solar production. Previous approaches to demand forecasting treated all types of customers equally. This work shows that a good forecasting system should treat customers of different type differently otherwise the system will experience more error.

% abstract.tex (Abstract)

\addcontentsline{toc}{chapter}{Abstract}

\chapter*{Abstract}

Accurate electricity load demand forecasting is an important problem in managing the power grid for both economic and environmental reasons. The Power TAC simulation provides a platform for research on smart grid energy generation and distribution systems. Brokers are the focus of the design task posed to developers by the system. The brokers work as self-interested entities that try to maximize profits by trading electricity across multiple markets. To be successful, a broker has to forecast  the electricity demand for customers as accurately as possible so it can use this information to operate efficiently. My proposed forecasting method uses a combination of clustering and classifiers. The customers are clustered based on a small history of weekly average load. After that, energy load history and weather related information are used as features to train classifiers for each cluster of customers. To forecast for a new customer, the proposed method needs at least one week of energy load history for the customer. The system assigns the  new customer to one of the clusters based on the similarity of its electricity usage history. The classifier for that cluster will be used to forecast the new customer. This approach produced 13\% error compared to 31\% relative absolute error observed for a moving average baseline predictor. Power TAC has six different types of customer such as customers with demand shifting capabilities, customers with no demand shifting capabilities, electric vehicles, thermal storage, wind production and solar production. Previous approaches to demand forecasting treated all types of customers equally. This work shows that a forecasting system that treats customers of different type differently by creating clusters of similar types can generalize effectively, having similar error rates to learning individual predictors for each cluster, while also allowing fast predictions for novel customers.

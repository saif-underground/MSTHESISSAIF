% abstract.tex (Abstract)

\addcontentsline{toc}{chapter}{Abstract}

\chapter*{Abstract}

Accurate electricity demand forecasting is an important problem as the failure to do so may be costly for both economic and environmental reasons. Power TAC simulation system provides a no risk platform to do research on smart grid based energy generation and distribution. Brokers are important components of the system. The brokers work as self-interested entities that try to maximize profits by trading electricity in various markets. To be successful, a broker has to forecast  the electricity demand about its customers as accurately as possible, otherwise it will operate ineffectively. This proposed forecasting method uses a combination of cluster and classifiers. At first, the customers are clustered based on their weekly average usage. After that, energy usage history and related weather related information are combined together to train classifier for the cluster. To forecast for a new customer, the proposed method needs at least a week's energy usage history of the customer. The system assigns the  new customer to one of the clusters based on its electricity usage history. The classifier for that cluster will be used to forecast the customer. This approach produced 13 \% error compared to 31\% relative absolute error observed against the moving average baseline predictor. The Power TAC system has six different types of customer such as customers with demand shifting capabilities, customer with no demand shifting capabilities, electric vehicles, thermal storage, wind production and solar production. Previous approaches to demand forecasting treated all types of customers equally. This work shows that a good forecasting system should treat customers of different type differently, otherwise the system will experience more error.

% chap2.tex (Definitions)

\chapter{Smart Grid and PowerTAC Competition}

In this chapter, I will describe Smart Grid and PowerTAC competition.

\section{Traditional Energy Distribution and Consumption System}
In traditional electricity generation system there are three subsystems \cite{fang2012smart}. In electricity generation subsystem, the generator rotates a turbine in magnetic field which generates electricity. The turbine rotates through the power of kinetic energy of water falling from a water fall or a river with strong current, or from the energy of nuclear powerplan or energy received from burning coal or oil. Traditional energy generation system then transmits the electricity through transmission grid and electricity gets distributed in the distribution grid. This generation system is one way meaning a single power generation source serves several consumption source.



\section{Smart Grid}
In contrast to the traditional electricity generation system, Smart Grid (SG) are two way \cite{fang2012smart}. So, any node in the distribution grid can produce electricity and push it to the distribuiton grid if necessary. The NIST report \cite{fang2012smart} states that the SG would make the electricity generation and supply robust against generator or distribution node failure, use renewable energy widely and efficiently, reduce green house gas emission, reduce oil consumption by encouraging usage of electric vehicles, it will give customers more freedom to choose among energy sources. Smart grids will encourage usage of electric vehicle as these vehicles have the ability to store power in a battery and transmit the power to the distribution grid if there is a necessity. The major challenge with the usage of renewable energy is it is uncertain. This uncertainity causes the ability to predict how much energy the SG can produce in a future time slot hard. Success of SG will need efficient methods to predict energy production \cite{potter2009building}.


\section{Smart Grid and Renewable Energy}
One of the major focus of Smart Grid(SG) will be using renewable energy. There are challenges involved with using this abundant source of energy \cite{richter2012transitioning}. People are already showing strong motivation to use renewable energy as indicated by the statistics that 20\% of total energy is from the renewable sources which is second after coal 24\%. People are using renewable energy due to economic reward and environmental concern. Major challenge with Renewable energy is amount of the energy produced is greatly varying. Since the energy produced is volatile there must be a storage mechanism that balances out the surplus energy. The usage of rechargeable electric vehicles might serve the purpose of storage. Accurate prediction of the renewable energy might enable the electric car users to absorb surplus energy and push it back to the grid in peak hours if necessary. 




\section{Importance of accurate load forecasting}.
Accurate load forecasting is important to ensure efficient fuel usage, reduce wastage of energy and planning proper operation of power generators \cite{liu2006accurate}.






%%% Local Variables: 
%%% mode: latex
%%% TeX-master: "thesis"
%%% End: 

% chap5.tex (Definitions, Theorem and Proof)

\chapter{Solving Narrow-Interval Linear Systems Is
NP-Hard: New Result} \label{NewResult}

In this chapter we present the main result upon which this thesis is
centered---a new theorem and a proof for it based upon the
reduction\footnote{The reduction used is called a {\em polynomial-time
one-one reduction}, a special case of polynomial-time many-one reductions.}
of a system of (arbitrary) interval linear equations to a system of
narrow-interval linear equations.

\section{Definitions}

\begin{definition}
{\rm We say that the (non-zero) number $\tilde x$ represents a number $x$ with
a {\em relative accuracy\/} $\delta>0$ if $\displaystyle{\frac{|x-\tilde x|}
{|\tilde x|}}\leq\delta$.}
\end{definition}

\begin{definition}
{\rm By {\em relative half-width\/} of the interval ${\bf x}=[x^-,x^+]$
where $0\not\in[x^-,x^+]$, we
mean the smallest number $\delta>0$ for which every number in ${\bf x}$ is
represented with a relative accuracy~$\delta$ by $\tilde x=\displaystyle{
\frac{x^-+x^+}{2}}$.\\[0.5pc]
{\bf Proposition} It can be seen that the relative half-width of ${\bf x}$
where $0\not\in[x^-,x^+]$ is}
$$
  \max_{x\in [x^-,x^+]}\frac{|x-\tilde x|}{|\tilde x|}=
   \frac{x^+-x^-}{2|\tilde x|}.
$$
\end{definition}

\begin{definition}
{\rm By {\em relative width\/} $W^{rel}$ of an interval ${\bf x}=[x^-,x^+]$
where $0\not\in[x^-,x^+]$,
we mean twice the relative half-width of ${\bf x}$, i.e.,
$$
  W^{rel}([x^-,x^+])=\frac{x^+-x^-}{|\tilde x|}.
$$}
\end{definition}

\begin{definition}
{\rm We say that an interval ${\bf x}$ is {\em $\delta$-narrow in the sense 
of relative accuracy\/} if $W^{rel}({\bf x})\leq\delta$.}
\end{definition}

\begin{definition}
{\rm We say that an interval ${\bf x}$ is {\em $\delta$-narrow\/} if it is
both $\delta$-narrow in the sense of absolute accuracy and $\delta$-narrow in
the sense of relative accuracy.}
\end{definition}

\section{Theorem}
\begin{theorem}
For every $\delta>0$, the problem of solving systems of interval linear
equations with $\delta$-narrow intervals is computationally intractable
(\/{\rm NP}-hard).
\end{theorem}

\begin{corollary}
For every $\Delta>0$, the problem of solving systems of interval linear
equations with intervals $\Delta$-narrow in the sense of absolute accuracy 
is computationally intractable (\/{\rm NP}-hard).
\end{corollary}

\begin{corollary}
For every $\delta>0$, the problem of solving systems of interval linear
equations with intervals $\delta$-narrow in the sense of relative accuracy is
computationally intractable (\/{\rm NP}-hard).
\end{corollary}

\newpage
\section{Proof}

\subsection{Part I: Reduction of Interval Linear System to Narrow-Interval
Linear System}

We have seen that the general problem of solving interval linear equation
systems (\ref{d1}) is NP-hard.  We now intend to show that this general
problem can be reduced to the problem of solving a system of 
interval linear equations with $\delta$-narrow intervals.

Let us start with an arbitrary interval linear equation system (\ref{d1}).  To
reduce it to a $\delta$-narrow interval linear equation system, we introduce
new variables $w_i$, $y_{ij}$ and $z_{ij}$ for all $i=1,\ldots,m$ and
$j=1,\ldots,n$.  For the enlarged list of $m+n+2mn$ variables (i.e., $x_j$,
$w_i$, $y_{ij}$ and $z_{ij}$) we introduce the following new system of
$m+n+2mn$ $\delta$-narrow interval linear equations\footnote{For clarity,
non-interval coefficients can be thought of as intervals of zero width; e.g., 
the coefficient for each variable $y_{ij}$ is $[1,1]$ and each coefficient 
$c_i$ is equivalent to the interval $[c_i,c_i]$.}:
\begin{eqnarray}
  \sum_{j=1}^n y_{ij} +
   \sum_{j=1}^n [1-\frac{\delta}{2},1+\frac{\delta}{2}]z_{ij} +
   [1-\frac{\delta}{2},1+\frac{\delta}{2}]w_i &=& c_i, \label{p1a}\\
  w_i &=& \gamma_i, \label{p1b}\\
  y_{ij} - \mu_{ij} x_j &=& 0, \label{p1c}\\
  z_{ij} - \nu_{ij} x_j &=& 0, \label{p1d}
\end{eqnarray}
where $\delta>0$ and the numerical (non-interval) coefficients $c_i$,
$\gamma_i\geq 0$, $\mu_{ij}$ and $\nu_{ij}\geq 0$ will be chosen in such a
way that for every solution of this $\delta$-narrow interval linear equation
system (\ref{p1a})--(\ref{p1d}) we have
\begin{equation}
  c_i - [1 - \frac{\delta}{2}, 1 + \frac{\delta}{2}] w_i =
   [b_i^-, b_i^+] \label{p2}
\end{equation}
and
\begin{equation}
  y_{ij} + [1 - \frac{\delta}{2}, 1 + \frac{\delta}{2}] z_{ij} =
   [a_{ij}^-, a_{ij}^+] x_j \label{p3}
\end{equation}
for all $i=1,\ldots,m$ and $j=1,\ldots,n$.

Let us first find the values for $c_i$ and $\gamma_i$.  From equation
(\ref{p1b}) we know that $w_i = \gamma_i$.  By substituting this expression
into equation (\ref{p2}) we get
$$
  c_i - [1 - \frac{\delta}{2}, 1 + \frac{\delta}{2}] \gamma_i =
   [b_i^-, b_i^+]. \label{p4}
$$
Intervals are equal {\em if and only if\/} their endpoints coincide.  Since we
are looking for a solution with $\gamma_i\geq 0$, the equations for the
endpoints take the following forms:
\begin{equation}
  c_i - (1 - \frac{\delta}{2}) \gamma_i = b_i^+ \label{p5}
\end{equation}
and
\begin{equation}
  c_i - (1 + \frac{\delta}{2}) \gamma_i = b_i^-. \label{p6}
\end{equation}
Subtracting the second equation from the first we get
$$
  \delta\cdot\gamma_i = b_i^+ - b_i^-,
$$
and hence,
\begin{equation}
  \fbox{${\displaystyle\gamma_i=\frac{1}{\delta}(b_i^+-b_i^-).}$} \label{GAMMA}
\end{equation}
Substituting this value into equation (\ref{p5}), we get
$$
  c_i - (1 - \frac{\delta}{2}) \frac{1}{\delta}(b_i^+ - b_i^-) = b_i^+
$$
from which we get
$$
  c_i - (\frac{1}{\delta} - \frac{1}{2}) (b_i^+ - b_i^-) =  b_i^+,
$$
and hence,
\begin{equation}
  \fbox{${\displaystyle c_i=\frac{1}{2}(b_i^++b_i^-)
   +\frac{1}{\delta}(b_i^+-b_i^-).}$}                                 \label{C}
\end{equation}

Now let us find the values for $\mu_{ij}$ and $\nu_{ij}$.  From
equations (\ref{p1c}) and (\ref{p1d}) we conclude that $y_{ij}=\mu_{ij}x_j$
and $z_{ij}=\nu_{ij}x_j$.  Substituting these expressions into equation
(\ref{p3}) we get
$$
  \mu_{ij} x_j + [1-\frac{\delta}{2},1+\frac{\delta}{2}] \nu_{ij} x_j =
   [a_{ij}^-, a_{ij}^+] x_j.
$$
For this equality to be true for all $x_j$, coefficients for $x_j$ on both
sides of the equation must coincide.  Thus, we conclude that
$$
  \mu_{ij} + [1 - \frac{\delta}{2}, 1 + \frac{\delta}{2}] \nu_{ij} =
   [a_{ij}^-, a_{ij}^+].
$$
As stated before, for intervals to be equal their endpoints must coincide,
and since we are looking for a solution for $\nu_{ij}\geq 0$, the equations
for the endpoints take the following forms:
\begin{equation}
  \mu_{ij} + (1 + \frac{\delta}{2}) \nu_{ij} = a_{ij}^+ \label{p8}
\end{equation}
and
\begin{equation}
  \mu_{ij} + (1 - \frac{\delta}{2}) \nu_{ij} = a_{ij}^-. \label{p9}
\end{equation}
Subtracting the second equation from the first we get
$$
  \delta\cdot\nu_{ij} = a_{ij}^+ - a_{ij}^-,
$$
and hence,
\begin{equation}
  \fbox{${\displaystyle\nu_{ij}=\frac{1}{\delta}(a_{ij}^+-a_{ij}^-).}$}
                                                                     \label{NU}
\end{equation}
Substituting this value back into equation (\ref{p8}) we get
$$
  \mu_{ij} + (1 + \frac{\delta}{2}) \frac{1}{\delta}(a_{ij}^+ - a_{ij}^-) =
   a_{ij}^+
$$
which leads to
$$
  \mu_{ij}+(\frac{1}{\delta}+\frac{1}{2})(a_{ij}^+ - a_{ij}^-) = a_{ij}^+,
$$
and hence,
\begin{equation}
  \fbox{${\displaystyle\mu_{ij}=\frac{1}{2}(a_{ij}^++a_{ij}^-)
   -\frac{1}{\delta}(a_{ij}^+-a_{ij}^-).}$}                          \label{MU}
\end{equation}

\newpage
\subsection{Part II: The Two Systems Are ``Equivalent''}

Let us show that every system (\ref{d1}) of interval linear equations is
``equivalent'' to the $\delta$-narrow interval linear equation system
(\ref{p1a})--(\ref{p1d}) in the following sense:
\begin{description}
  \item[Claim 1] If $(x_1,\ldots,x_n)$ is a solution of the original system
    (\ref{d1}), then for the same values $x_1,\ldots,x_n$, and for some values
    $w_1,\ldots,w_m$, $y_{11},\ldots,y_{mn}$ and $z_{11},\ldots,z_{mn}$, the 
    extended tuple
    \begin{equation}
       (x_1,\ldots,x_n,w_1,\ldots,w_m,y_{1 1},\ldots,y_{m n},
        z_{1 1},\ldots,z_{m n}) \label{extsolution}
    \end{equation}
    is a solution of the $\delta$-narrow interval system
    (\ref{p1a})--(\ref{p1d}).
  \item[Claim 2] Conversely, if tuple (\ref{extsolution}) is a solution of the
    $\delta$-narrow interval linear equation system (\ref{p1a})--(\ref{p1d}),
    then $(x_1,\ldots,x_n)$ is a solution of the original system (\ref{d1})
    for the same values $x_1,\ldots,x_n$.
\end{description}
We will now prove that the two systems are indeed ``equivalent'' in the above
sense.

\subsubsection{Proof of Claim 1}
Let us assume that ($x_1,\ldots,x_n$) is a solution of the
original system (\ref{d1}).  By definition, 
if $(x_1,\ldots,x_n)$ is a solution to the original system (\ref{d1}), then
there must exist values $a_{ij}\in[a_{ij}^-,a_{ij}^+]$ and
$b_i\in[b_i^-,b_i^+]$ such that
\begin{equation}
  \sum_{j=1}^n a_{ij} x_j = b_i \label{claim1}
\end{equation}
for all $i=1,\ldots,m$.  With this in mind, let us introduce new parameters
$\alpha_{ij}=a_{ij}-a_{ij}^-$ and $\beta_i=b_i-b_i^-$; thus, $a_{ij}=a_{ij}^-+
\alpha_{ij}$ and $b_i=b_i^-+\beta_i$.  Substituting these expressions into
(\ref{claim1}) we get
\begin{equation}
  \sum_{j=1}^n (a_{ij}^-+\alpha_{ij}) x_j = b_i^-+\beta_i. \label{newsolution}
\end{equation}
Since $a_{ij}\in[a_{ij}^-,a_{ij}^+]$, we conclude that 
$\alpha_{ij}\in[0,a_{ij}^+-a_{ij}^-]$.  Similarly, since
$b_i\in[b_i^-,b_i^+]$, we conclude that $\beta_i\in[0,b_i^+-b_i^-]$.

Now let us show that there exists an extended tuple (\ref{extsolution}) that is
a solution to system (\ref{p1a})--(\ref{p1d}) for the same values
$x_1,\ldots,x_n$ and for appropriately chosen values of $w_1,\ldots,w_m$,
$y_{11},\ldots,y_{mn}$ and $z_{11},\ldots,z_{mn}$.  In order to have equations
(\ref{p1b})--(\ref{p1d}) automatically satisfied, we will take $w_i=\gamma_i$,
$y_{ij}=\mu_{ij}x_j$ and $z_{ij}=\nu_{ij}x_j$.  It is thus sufficient to show
that equation (\ref{p1a}) is satisfied for all $i=1,\ldots,m$; in other words,
we need to show that there exist values
${\displaystyle\kappa_i\in[1-\frac{\delta}{2},1+\frac{\delta}{2}]}$ and
${\displaystyle\lambda_{ij}\in[1-\frac{\delta}{2},1+\frac{\delta}{2}]}$ such
that
\begin{equation}
  \sum_{j=1}^n y_{ij}+\sum_{j=1}^n\lambda_{ij}z_{ij}+\kappa_i w_i=c_i
   \label{s1a}
\end{equation}
for all $i=1,\ldots,m$.

Note that for any given $i$, if $\gamma_i=0$ (and thus $w_i=0$), then
$\kappa_i w_i=0$, implying that $\kappa_i$ can be {\em any value\/} as long as
${\displaystyle\kappa_i\in[1-\frac{\delta}{2},1+\frac{\delta}{2}]}$ (e.g., 
$\kappa_i=1$).  Likewise, for any given $i$ and $j$, if $\nu_{ij}=0$ (and thus 
$z_{ij}=0$), then  $\lambda_{ij}z_{ij}=0$, implying that $\lambda_{ij}$ can be 
{\em any value\/} as long as ${\displaystyle\lambda_{ij}\in[1-\frac{\delta}{2},
1+\frac{\delta}{2}]}$ (e.g., $\lambda_{ij}=1$).  Therefore, for the remainder
of the proof we will be choosing $\kappa_i$ only for those $i$ for which
$\gamma_i>0$ (i.e., $b_i^-<b_i^+$), and we will be choosing $\lambda_{ij}$ 
only for those $i$ and $j$ for which $\nu_{ij}>0$ (i.e., $a_{ij}^-<a_{ij}^+$).

We have chosen $w_i$ and $c_i$ for which the equation (\ref{p2}) is true,
i.e., for which
$$
  c_i - [1 - \frac{\delta}{2}, 1 + \frac{\delta}{2}] w_i = [b_i^-, b_i^+].
$$
Crudely speaking, this equality means that when we take different values for
$\kappa_i$ such that
$\kappa_i\in[1-{\displaystyle\frac{\delta}{2}},1+{\displaystyle\frac{\delta}
{2}}]$, the set
$$
  c_i - [1 - \frac{\delta}{2}, 1 + \frac{\delta}{2}] w_i
$$
of possible values of $c_i-\kappa_i w_i$ coincides with the interval
$[b_i^-, b_i^+]$.  In particular, since $b_i\in[b_i^-,b_i^+]$, there must
exist a value $\kappa_i\in[1-{\displaystyle\frac{\delta}{2}},1+{\displaystyle
\frac{\delta}{2}}]$ for which
$$
  c_i-\kappa_i w_i = b_i = b_i^- + \beta_i.
$$
From this equation we can find the exact value of $\kappa_i$.

We know from equation (\ref{p1b}) that $w_i=\gamma_i$, so by substitution we
get
$$
  c_i - \kappa_i \gamma_i = b_i^- + \beta_i,
$$
and hence,
$$
  \kappa_i=\frac{1}{\gamma_i}(c_i-b_i^--\beta_i).
$$
Substituting the values of $\gamma_i$ and $c_i$ from equations (\ref{GAMMA})
and (\ref{C}) we get
$$
  \kappa_i=\frac{\delta}{b_i^+-b_i^-}(\frac{1}{2}(b_i^++b_i^-) +
   \frac{1}{\delta}(b_i^+-b_i^-) - b_i^- - \beta_i).
$$
Simplifying, we get
$$
  \kappa_i=1+\frac{\delta}{b_i^+-b_i^-}(\frac{1}{2}(b_i^++b_i^-)-b_i^--\beta_i)
$$
from which we get
$$
  \kappa_i=1+\frac{\delta}{b_i^+-b_i^-}(\frac{1}{2}(b_i^+-b_i^-)-\beta_i),
$$
and hence,
\begin{equation}
  \fbox{${\displaystyle\kappa_i=1+\frac{\delta}{2}-
   \frac{\delta\cdot\beta_i}{b_i^+-b_i^-}.}$}                     \label{KAPPA}
\end{equation}

We have also chosen $y_{ij}$ and $z_{ij}$ for which the equation (\ref{p3}) is
true, i.e., for which
$$
  y_{ij} + [1 - \frac{\delta}{2}, 1 + \frac{\delta}{2}] z_{ij} =
   [a_{ij}^-, a_{ij}^+] x_j
$$
Crudely speaking, this equality means that when we take different values for
$\lambda_{ij}$ such that $\lambda_{ij}\in[1-{\displaystyle\frac{\delta}{2}},
1+{\displaystyle\frac{\delta}{2}}]$, the set
$$
  y_{ij} + [1 - \frac{\delta}{2}, 1 + \frac{\delta}{2}] z_{ij}
$$
of possible values of $y_{ij}+\lambda_{ij}z_{ij}$ coincides with the interval
$[a_{ij}^-, a_{ij}^+]x_j$. In particular, since $a_{ij}\in[a_{ij}^-,a_{ij}^+]$,
there must exist a value $\lambda_{ij}\in
[1-{\displaystyle\frac{\delta}{2}},1+{\displaystyle\frac{\delta}{2}}]$ for
which
$$
  y_{ij} + \lambda_{ij} z_{ij} = (a_{ij}^- + \alpha_{ij}) x_j.
$$
From this equation we can find the exact value of $\lambda_{ij}$.

We conclude from equations (\ref{p1c}) and (\ref{p1d}) that
$y_{ij}=\mu_{ij}x_j$ and $z_{ij}=\nu_{ij}x_j$, so by substitution we get
$$
  \mu_{ij} x_j + \lambda_{ij} \nu_{ij} x_j = (a_{ij}^- + \alpha_{ij}) x_j,
$$
and hence,
$$
  \lambda_{ij}=\frac{1}{\nu_{ij}}(a_{ij}^-+\alpha_{ij}-\mu_{ij}).
$$
Substituting the values of the coefficients $\nu_{ij}$ and $\mu_{ij}$ from
equations (\ref{NU}) and (\ref{MU}) we get
$$
  \lambda_{ij}=\frac{\delta}{a_{ij}^+-a_{ij}^-}(a_{ij}^-+\alpha_{ij}-
   \frac{1}{2}(a_{ij}^++a_{ij}^-) + \frac{1}{\delta}(a_{ij}^+-a_{ij}^-)).
$$
Simplifying, we get
$$
  \lambda_{ij}=1+\frac{\delta}{a_{ij}^+-a_{ij}^-}
   (a_{ij}^-+\alpha_{ij}-\frac{1}{2}(a_{ij}^++a_{ij}^-))
$$
from which we get
$$
  \lambda_{ij}=1-\frac{\delta}{a_{ij}^+-a_{ij}^-}
   (\frac{1}{2}(a_{ij}^+-a_{ij}^-)-\alpha_{ij}),
$$
and hence,
\begin{equation}
  \fbox{${\displaystyle\lambda_{ij}=1-\frac{\delta}{2}+
   \frac{\delta\cdot\alpha_{ij}}{a_{ij}^+-a_{ij}^-}.}$}          \label{LAMBDA}
\end{equation}

To show that these values for $\kappa_i$ and $\lambda_{ij}$ are indeed the
ones desired, we first note that, since $\beta_i\in[0,b_i^+-b_i^-]$ and
$\alpha_{ij}\in[0,a_{ij}^+-a_{ij}^-]$, we can conclude that
$$
  \frac{\beta_i}{b_i^+-b_i^-}\in[0,1]
$$
and
$$
  \frac{\alpha_{ij}}{a_{ij}^+-a_{ij}^-}\in[0,1].
$$
Thus,
$$
  \frac{\delta\cdot\beta_i}{b_i^+-b_i^-}\in[0,\delta]
$$
and
$$
  \frac{\delta\cdot\alpha_{ij}}{a_{ij}^+-a_{ij}^-}\in[0,\delta]
$$
from which it follows that
$$
  1+\frac{\delta}{2}-\frac{\delta\cdot\beta_i}{b_i^+-b_i^-}
   \in[1-\frac{\delta}{2},1+\frac{\delta}{2}]
$$
and
$$
  1-\frac{\delta}{2}+\frac{\delta\cdot\alpha_{ij}}{a_{ij}^+-a_{ij}^-}
   \in[1-\frac{\delta}{2},1+\frac{\delta}{2}],
$$
and hence,
$$
  \kappa_i\in[1-\frac{\delta}{2},1+\frac{\delta}{2}]
$$
and
$$
  \lambda_{ij}\in[1-\frac{\delta}{2},1+\frac{\delta}{2}].
$$

Our goal is to prove that equation (\ref{p1a}) is satisfied.  By showing that
we have found appropriate values for $\kappa_i$ and $\lambda_{ij}$ such that
equation (\ref{s1a}) is satisfied, we can do just that.  We have already
chosen the values for $y_{ij}$ and $z_{ij}$.  Substituting these values into
equation (\ref{s1a}) we get an equivalent equation
$$
  \sum_{j=1}^n \mu_{ij} x_j + \sum_{j=1}^n \lambda_{ij} \nu_{ij} x_j +
   \kappa_i w_i = c_i.
$$
This equation, in its turn, is equivalent to the equation
\begin{equation}
  \sum_{j=1}^n(\mu_{ij}+\lambda_{ij}\nu_{ij})x_j=c_i-\kappa_i w_i. \label{s1x}
\end{equation}

Let us now show that this equation is satisfied and thus prove that the
equivalent equation (\ref{s1a}) is satisfied as well.
\begin{itemize}
\item According to our choice of $\kappa_i$, the right-hand side of equation
  (\ref{s1x}) is equal to $b_i$.
\item According to our choice of $\lambda_{ij}$, we have
  $$
    \mu_{ij}+\lambda_{ij}\nu_{ij}=a_{ij},
  $$
  and hence, the left-hand side of equation (\ref{s1x}) is equal to
  $$
    \sum_{j=1}^n a_{ij}x_j.
  $$
\end{itemize}
We started with $x_1,\ldots,x_n$ for which ${\displaystyle\sum_{j=1}^n a_{ij}
x_j=b_i}$; hence, the left-hand side and the right-hand side of equation
(\ref{s1x}) coincide.  Thus, we have proven that equation (\ref{s1a}) is
satisfied.

Since we are able to show that there do indeed exist values
${\displaystyle\kappa_i\in[1-\frac{\delta}{2},1+\frac{\delta}{2}]}$ and
${\displaystyle\lambda_{ij}\in[1-\frac{\delta}{2},1+\frac{\delta}{2}]}$ such
that equation (\ref{s1a}) is satisfied for all $i=1,\ldots,m$, it follows that
if $(x_1,\ldots,x_n)$ is a solution of system (\ref{d1}), then for the same
values $x_1,\ldots,x_n$ and for some values $w_1,\ldots,w_m$,
$y_{11},\ldots,y_{mn}$ and $z_{11},\ldots,z_{mn}$, there exists an extended
tuple (\ref{extsolution}) that is a solution of system
(\ref{p1a})--(\ref{p1d}).  This proves Claim~1.

\subsubsection{Proof of Claim 2}
Now we must show that the converse is true, namely, that if tuple
(\ref{extsolution}) is a solution of the $\delta$-narrow interval linear
equation system (\ref{p1a})--(\ref{p1d}), then $(x_1,\ldots,x_n)$ is a 
solution of the interval linear equation system (\ref{d1}) for the same 
values $x_1,\ldots,x_n$.

Let us assume tuple (\ref{extsolution}) is a solution of system
(\ref{p1a})--(\ref{p1d}).  For this to be true, equations
(\ref{p1b})--(\ref{p1d}) must be satisfied, and thus we can conclude that
$w_i=\gamma_i$, $y_{ij}=\mu_{ij}x_j$ and $z_{ij}=\nu_{ij}x_j$.
The fact that equation (\ref{p1a}) is satisfied by a given tuple means that
\begin{equation}
  \sum_{j=1}^n y_{ij} + \sum_{j=1}^n \lambda_{ij}z_{ij} + \kappa_i w_i = c_i
   \label{x1}
\end{equation}
for some $\lambda_{ij}\in[1-{\displaystyle\frac{\delta}{2}},1+{\displaystyle
\frac{\delta}{2}}]$ and $\kappa_i\in[1-{\displaystyle\frac{\delta}{2}},1+
{\displaystyle\frac{\delta}{2}}]$.  Let us show that
\begin{equation}
  \sum_{j=1}^n a_{ij}x_j = b_i \label{x1.5}
\end{equation}
for some values $a_{ij}\in[a_{ij}^-,a_{ij}^+]$ and $b_i\in[b_i^-,b_i^+]$.

From equation (\ref{x1}) we conclude that
$$
  \sum_{j=1}^n y_{ij} + \sum_{j=1}^n \lambda_{ij}z_{ij} = c_i - \kappa_i w_i.
$$
Since $w_i=\gamma_i$, $y_{ij}=\mu_{ij}x_j$ and $z_{ij}=\nu_{ij}x_j$, by
substitution we conclude that
$$
  \sum_{j=1}^n (\mu_{ij} + \lambda_{ij}\nu_{ij})x_j = c_i - \kappa_i\gamma_i.
$$
Note that this equation has the desired form (\ref{x1.5}) where
\begin{equation}
  a_{ij}=\mu_{ij}+\lambda_{ij}\nu_{ij}                                \label{A}
\end{equation}
and 
\begin{equation}
  b_i=c_i-\kappa_i\gamma_i.                                           \label{B}
\end{equation}
Thus, to complete the proof, we need only show that 
$a_{ij}\in[a_{ij}^-,a_{ij}^+]$ and $b_i\in[b_i^-,b_i^+]$.

Let us first show that $b_i\in[b_i^-,b_i^+]$.  Since $\kappa_i\in[1-
{\displaystyle\frac{\delta}{2}},1+{\displaystyle\frac{\delta}{2}}]$, we have
that
$$
  1+\frac{\delta}{2} \geq \kappa_i \geq 1-\frac{\delta}{2}.
$$
By multiplying all sides of the inequality by $\gamma_i\geq 0$, we conclude
that
$$
  (1+\frac{\delta}{2})\gamma_i \geq \kappa_i\gamma_i \geq
   (1-\frac{\delta}{2})\gamma_i.
$$
By subtracting all parts of this inequality from $c_i$, we conclude that
$$
  c_i-(1+\frac{\delta}{2})\gamma_i \leq c_i-\kappa_i\gamma_i \leq
   c_i-(1-\frac{\delta}{2})\gamma_i.
$$
According to our choice of $c_i$ and $\gamma_i$ (which led to equations 
(\ref{p5}) and (\ref{p6})), this is equivalent to
$$
  b_i^- \leq c_i-\kappa_i\gamma_i \leq b_i^+.
$$
Thus, the value $b_i=c_i-\kappa_i\gamma_i$ we have chosen indeed belongs
to the interval $[b_i^-,b_i^+]$.

Let us now show, in like fashion, that $a_{ij}\in[a_{ij}^-,a_{ij}^+]$.  Since
$\lambda_{ij}\in[1-{\displaystyle\frac{\delta}{2}},1+{\displaystyle\frac
{\delta}{2}}]$, we have that
$$
  1-\frac{\delta}{2} \leq \lambda_{ij} \leq 1+\frac{\delta}{2}.
$$
By multiplying all sides of the inequality by $\nu_{ij}\geq 0$, we conclude
that
$$
  (1-\frac{\delta}{2})\nu_{ij} \leq \lambda_{ij}\nu_{ij} \leq
   (1+\frac{\delta}{2})\nu_{ij}.
$$
By adding all parts of this inequality to $\mu_{ij}$, we conclude that
$$
  \mu_{ij}+(1-\frac{\delta}{2})\nu_{ij}\leq\mu_{ij}+\lambda_{ij}\nu_{ij} \leq
   \mu_{ij}+(1+\frac{\delta}{2})\nu_{ij}.
$$
According to our choice of $\mu_{ij}$ and $\nu_{ij}$ (which led to equations
(\ref{p8}) and (\ref{p9})), this is equivalent to
$$
  a_{ij}^- \leq \mu_{ij}-\lambda_{ij}\nu_{ij} \leq a_{ij}^+,
$$
i.e., the value $a_{ij}=\mu_{ij}-\lambda_{ij}\nu_{ij}$ that we have chosen
indeed belongs to the interval $[a_{ij}^-,a_{ij}^+]$.

Thus, we have shown that if tuple (\ref{extsolution}) is a solution of system
(\ref{p1a})--(\ref{p1d}), then $(x_1,\ldots,x_n)$ is a solution of system
(\ref{d1}) for the same values $x_1,\ldots,x_n$.  This proves Claim~2.

\newpage
\subsection{Conclusion}

\noindent
We saw in part~II that the two systems, namely, the interval linear
equation system (\ref{d1}) and the $\delta$-narrow interval linear equation
system (\ref{p1a})--(\ref{p1d}), are ``equivalent'' in the sense described
therein.  In part~I we saw that the number of equations and unknowns of system
(\ref{p1a})--(\ref{p1d}) is bounded by a polynomial of the number of equations
and unknowns of system (\ref{d1}).  Further, the number of steps required for 
reducing an interval linear equation system (\ref{d1}) to a $\delta$-narrow 
interval linear equation system (\ref{p1a})--(\ref{p1d}) is bounded by a 
polynomial of the number of equations and unknowns of system (\ref{d1}).  
Thus, it follows from parts I and II that if we have an algorithm $\cal U$ 
that can solve $\delta$-narrow interval linear equation systems in polynomial 
time, then we can solve an arbitrary
system (\ref{d1}) of interval linear equations in polynomial time by applying
this hypothetical algorithm $\cal U$ to the ``equivalent'' $\delta$-narrow
interval linear equation system (\ref{p1a})--(\ref{p1d}).  But, as stated in
part~I, solving interval linear equation systems is an NP-hard problem.

So, if we can (in general) solve interval linear equation systems in
polynomial time, we can solve any problem from the class NP in polynomial time.
Therefore, if we can (in general) solve $\delta$-narrow interval linear
equation systems in polynomial time, we can solve any problem from the class
NP in polynomial time.  Thus, we conclude that {\em the problem of solving
$\delta$-narrow interval linear equation systems is\/ {\rm NP}-hard}. Q.E.D.

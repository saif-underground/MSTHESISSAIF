% chap6.tex (Conclusions and Future Work)

\chapter{Conclusions and Future Work}

\section{Significance of the Result}
This work showed that a single demand predictor may not be suitable for the Power TAC scenario. Also, it was clear that for each power type, a broker should use different demand forecasting mechanism. For consumption type customers a small set of features containing temperature, cloud cover, windspeed, average electricity usage and standard deviation of electricity usage produced good results. It was evident that cluster based demand prediction mechanism was a highly dependable demand prediction scheme. We also saw that for consumption type customers in the Power TAC simulation, the size of the cluster does not matter. The individual predictor scheme can be considered as a mechanism with n clusters, where n is the number of consumption customers. The proposed forecasting methodology was able to achieve almost similar demand forecasting accuracy using only 4 clusters instead of n clusters. The individual predictor scheme has a serious fault, the demand predictors are hardcoded by the customer names. If during the simulation the name of a customer is changed, this mechanism will not work. On the other hand, the proposed mechanism can be trained on previous game logs and does not have the problem mentioned above. Finally, the accuracy of the model may not depend on the size of the training set if customer properties are not changed in server.

\section{Future Work}

This work only deals with demand forecasting about the consumption customers. The proposed mechanism seems to be applicable for solar energy production customers with a slight change. For customer with irregular demand pattern such as customers with demand shifting capabilities and the electric vehicle customers, different technique of demand forecasting has to be figured out. Also, observing customer's demand shifting capabilities, more works can be done by shifting demands of the customers in a preferable off peak hour to gain more profit.

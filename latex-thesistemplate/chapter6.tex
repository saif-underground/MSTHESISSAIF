% chap6.tex (Conclusions and Future Work)

\chapter{Conclusions and Future Work}

My work demonstrates that a single demand predictor may not be suitable for the Power TAC scenario. It was clear that for each power type, a broker should use different demand forecasting mechanism. For consumption type customers a small set of features containing temperature, cloud cover, wind speed, average electricity usage and standard deviation of electricity usage produced good results. I provided evidence that cluster based demand prediction mechanism was a highly dependable demand prediction scheme. I also showed that for consumption type customers in the Power TAC simulation, the size of the cluster does not matter when size of the cluster is greater than or equal 4. The individual predictor scheme can be considered as a forecasting mechanism with 17 clusters. The proposed forecasting methodology was able to achieve very similar demand forecasting accuracy using only 4 clusters instead of 17 clusters. The individual predictor scheme has a serious weakness, in fact the demand predictors are hardcoded by the customer names. If during the simulation the name of a customer is changed, this mechanism will not work. On the other hand, the proposed mechanism can be trained on previous game logs and does not have the problem with handling new customers. Finally, I showed the accuracy of the model may not depend on the size of the training set if customer properties are not changed in server.

\section{Future Work}

My work so far only deals with demand forecasting for the consumption customers. I would like to compare the proposed method against other baselines. Instead of using simulation data, real-world data can be used to test the effectiveness of the proposed method. It may be the case that there are more helpful features out there, I would like to do more experiment with new features. Instead of using simple classifiers, I would like to train complex classifiers such as Neural Networks \cite{witten2005data} and Support Vector Machines \cite{witten2005data} to see if the accuracy can be improved more.  The proposed mechanism seems to be applicable for solar energy production customers with a slight change. For customer with irregular demand pattern such as customers with demand shifting capabilities and the electric vehicle customers, different technique of demand forecasting has to be figured out. 

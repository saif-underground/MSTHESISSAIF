% chap3.tex (Definitions and Theorem)

\chapter{Related Works}

There are mainly two types of load forecasting namely short term load forecasting and long term load forecasting. Short term load forecasting deals with forecasting up to a couple of weeks. Long term load forecasting may forecast customer's demand over month or year \cite{cho1995customer}. In this chapter, I have described different methods of energy load forecasting for long term and short term in the literature.

\section{Variables in Electricity Demand}

Studies such as \cite{hor2005analyzing}, \cite{hart2004weather} and \cite{cho1995customer} have found that temperature has effect on electricity demand. The study in \cite{hart2004weather} was done in a region of Australia.  It was found that,  in a lower temperature the customers tend to use heaters and in a higher temperature they tend to use coolers. As a result, the increase or decrease of temperature from a certain point will cause the consumption of electricity to increase. In study \cite{cho1995customer}, two demand forecasting models were proposed. One was univariate Auto Regressive Integrated Moving Average (ARIMA) and another one was univariate ARIMA model along with temperature depended transfer function. The model with temperature variable did better forecasting than the one without the temperature variable. On the other hand, the study in  \cite{chen2004load} showed that inclusion of temperature variable in forecasting model actually introduced more error in demand prediction. The aim of the study was to make forecasting about electricity usage of January based on past five years training data using a Support Vector Machine (SVM)  forecaster. The reason behind of getting more error after including temperature variable may be because during Januray the temperature did not change much and the inclusion might have caused overfitting. 

Weather variables such as  wind speed and cloud cover has effect on electricity demand \cite{hor2005analyzing}, \cite{rudenauer2004energy}. As cloud cover increases, the demand for light increases too. The increased lighting demand causes increased electricity demand. The period where cloud cover was low, the electricity demand was also low \cite{hor2005analyzing}. High speed wind across wet walls help cool houses. High speed wind thus may cause reduced electricity demand due to reduced demand of air cooling \cite{rudenauer2004energy}. 


In the survey article \cite{hahn2009electric}, the authors reported that the day of the week and the month of the year is highly correlated with customer's energy demand. The electricity load demand can be higher or lower based on the day of week. The weekends usually have different load demand pattern than the week days. Also, based on the hour of a given day, the load demand can be higher or lower too. The season also showed impact on electricity demand. 

\section{Electricity Load Forecasting Using Statistical Method}

To make electricity load forecast, researchers have used statistical methods such as statistical average and Auto Regressive Integrated Moving Average (ARIMA).  Agent TACTEX'13, the winner of the PowerTAC competition in 2013, used the statistical average to make electricity demand forecasting for an hour of a day of a week. In a week a customer has 24 * 7 = 168 hours or slots where it can consume electricity. TACTEX'13  kept track of average usage of 168 weekly slots for each customer. To predict a future time slot, their agent would look at which weekly slot the future time slot would fall in. Then the agent used that weekly slot's average usage as the forecast of the future time slot. \cite{cho1995customer} have used an ARIMA model for load forecasting. The ARIMA model uses both moving average and auto regression to forecast the demand. To make a forecast about a future time slot, the auto regression model uses some previously observed time slots values based on its degree. Moving average scheme would use the average of all the known time series data points to make a prediction about a future time slot. In the study \cite{amjady2001short}, a short term load demand was proposed that uses several ARIMA models. For the combination of week day and temperature level, 16 short term load forecasting models were used. This scheme made better forecasting than a single ARIMA model.     


\section{Load Forecasting using Machine Learning}
\cite{parra2013initial} the authors used various machine learning techniques to make 24 hours ahead load forecast for the Power TAC simulation. They found that hour of week, weather related features such as temperature cloud cover were influential to the electricity load demand. They created one machine learning electricity load forecasting module for each customer by extracting relevant features for the customers. The forecasting modules performed well for the customers that showed regularity in their energy consumption behavior. For the customers with load shifting capabilities to their favored hour and customers with irregular consumption patterns, the scheme did no perform well.

\section{Load Forecasting for a Specific Region}
Regional load forecasting will enable the authority to know which regions need more energy. If they know which regions need more energy, they will know most suitable places to place electricity generator plants. \cite{hsu2003regional} worked on load forecasting based on region. They divided electricity usage of Taiwan into 4 areas. For each region, they collected GDP,  population, highest temperature and aggregated load. After that, they trained Artificial Neural Network (ANN) model for each region. For the baseline, they trained linear regression model for each region. The result showed that the ANN-based load forecasting methods performed better than the linear regression  methods.


\section {Load Forecasting using Clustering}
%talk about clustering for load forecasting
\cite{mcloughlin2015clustering} have used the clustering method to forecast customer's future electricity demand. They collected data from more than 4000 household customers in Ireland for about 6 months. Collected data included electricity usage at 30 minutes interval, appliances used in the home and different socio-economic information about the people living in a particular house. They clustered each day's usage which they call load profiles. A customer's daily usage then can be assigned to one of those load profiles. The customer is then characterized by the mostly used load profile. The authors then trained a linear regression classifier that was built upon the socio-economic information of the customers, types of appliances used in the house and the description of the house to figure out the common load profile of the given household. The predicted load profile of the customer received from the linear regression model will be used to predict the demand of the customer for a given day. \cite{cho1995customer} noticed difference of behavior among customers. They manually clustered the population into four categories namely commercial, office, residential and industrial customers. In their paper \cite{wang2015gongbroker}, the authors proposed a novel demand prediction mechanism. In Power TAC competition, every broker is provided with past two weeks usage of all the customers or bootstrap usage. Their proposed broker clustered the customers based on the bootstrap data. For each cluster, the broker would make a linear regression model. The input variables included past average usage and weather related information. This approach of prediction clusters is based on the usage pattern of the customers. So this method may not be suitable for customers with irregular usage pattern such as customers with load shifting capabilities and electric vehicle customers. 

\section{Expert System based Load Forecasting}
%expert systems for load forecasting
The authors in ref \cite{rahman1988expert} have proposed an expert system based load forecasting method for the region, Virgina. The expert system would forecast the load of upcoming 24 hours. They observed the variables that are likely to affect the load. They came up with variables such as temperature, load of the previous hour, season, and day of the week have a strong correlation with the observed load. They implemented a computer program that mimicked how a human operator makes load forecast based on the independent variables. For a specific region's weather condition, their method worked well and required a limited amount of historical data. 


From the review of the literature, the importance of weather related variables such as temperature, cloud cover and wind speed is evident. Also, the hour of the day and day of the week are highly correlated with the load demand. A combination of machine learning classifiers and clustering algorithms appears to be a better idea. For the methodology of \cite{parra2013initial} it will take a large number of predictors for the simulation system. Also, those predictors will not work if the name of the customer is changed or a new customer is introduced as each predictor is hard coded with a specific customer. It sounds reasonable to cluster the data first and then train machine learning classifier for each cluster. This approach will hold generality. Instead of training only on bootstrap data as the \cite{wang2015gongbroker} have done, a wealth of data generated from the simulations can be used to  train the cluster. Since the clustering is done offline, this approach will not  suffer from the problem of having a time limit that the broker has to face if the cluster is trained during the competition. After the clustering is done, for each cluster, different machine learning classifiers can be trained to figure out which one performs the best. So, the broker will no longer stick to linear regression. This way, the training module will be able to deal with new customers.


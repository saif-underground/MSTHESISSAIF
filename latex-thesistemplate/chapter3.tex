% chap3.tex (Definitions and Theorem)

\chapter{Related Works}


In this chapter I havel described different methods of energy load forecasting in the literature. 

\section{types of load forecasting}
%types of load forecasting
There are mainly two types of load forecasting namely short term load forecasting and long term load forecasting. Short term load forecasting deals with forecasting upto couple of weeks. Long term load forecasting may forecast customer's demand over month or year \cite{cho1995customer}.

\section{variables in energy demand}
Customer's energy demand is correlated with the weather variables such as temperature and the day of the week. Researchers have found that weather effect relies on the time duration the training data has. \cite{chen2004load} trained a SVM energy demand predictor that would predict energy demand of customers for the month January. The training data consited of every half hour's electricity demand from 1997 to 1998, average temperature from 1995 to 1998. They trained the predictor with only the portion of data that are related to the month January. They have found that within the month of January the temperature does not vary much and excluding the temperature from the feature set actually gives better prediction. Again, if energy demand is long term which means the window of prediciton is about a year, the temperature seems to have effect on the energy demand of the customers. \cite{hart2004weather} collected data of 18 months from households of a region of Australia. They collected the weather data from weather office and from self transplanted devices. They observed how the household customers use appliances based on the temperature. They came into conclusion that for that region, equilibrium point for energy usage is at temperature 0.25 degree celcius. If the temperature increases or decreases from this temperature, the electricity usage increases. They explained the behavior by stating as the temperature decreases, houses customers tend to use heaters and if the temperature rises they tend to use coolers more. \cite{cho1995customer} proposed a model that used a transfer function that relates the daily temperature with energy usage along with the ARIMA model. This scheme resulted better than the univariate ARIMA model. 
%effect of variables 
In the survey article \cite{hahn2009electric}, the authors reported that the day of week and the month of year is highly correlated with customer's energy demand. They have found that based on the hour of a given the load demand can be higher or lower. They have also found that the weekends usually have different load demand than the usual days. Finally, they found that customers load demand changes based on the season of the year. They have concluded that, weather related variables, seasonal variables should be included in the long term prediction models .


\section{load prediction using statistical method}

To make load forecast researchers have used statistical methods such as statistical average and Auto Regressive Integrated Moving Average (ARIMA).  Agent TACTEX'13, the winner of the PowerTAC competition in 2013 used statistical average to make prediction for an hour of a day of a week. In a week a customer have 24 * 7 = 168 hours or slots. TACTEX'13  kept track of average usage of 168 weekly slot for each customer. To predict a future time slot, their agent would look at at which weekly slot the future time slot would fall in. Then the agent used that weekly slot's average usage as the prediciton of the future slot. \cite{cho1995customer} have used ARIMA model for load forecasting. The ARIMA model uses both moving average and auto regression to forecast the demand. To make a forecast about a future time slot, the auto regression model uses some previously observed time slots values based on its degree. Moving average scheme would use the average of all the known time series data points to make a prediciton about a future time slot .Problem with univariate ARIMA model is that they don't take into account other variables that my affect the demand such as temperature.   


\section{load prediction using machine learning}
\cite{parra2013initial} the authors used varios machine learning techniques to make 24 hour ahead load forecast. They found that hour of week, weather related features such as temperature cloud cover were influential to the electricity load. They created one machine learnign forecasting module for each customers by extracting relevant features of the customers. The forecasting modules performed well for the customers that shows regularity in their energy consumption behavior. For the customers with load shifting capabilities to their favored hour, the scheme did no perform well.

\section{load prediction for specific region}
Regional load forecasting will enable us to know which regions need more energy. If we know which regions need more energy, we will know most suitable places to place electricity generator plants. \cite{hsu2003regional} worked on load forecasting based on region. They diivided electricity usage of Taiwan in 4 areas. For each region, they collected GDP,  population, highest temperature and aggregated load. After that, they trained Artificial Neural Network model for each region. For baseline, they trained linear regression model for each region. The result showed that, the ANN based load forecasting methods performed better than the linear regression  methods. 


\section {load prediction using clustering}
%talk about clustering for load forecasting
\cite{mcloughlin2015clustering} have used clustering method to forecast customer's future electricity demand. They collected data from more than 4000 household customers in Ireland for about 6 months. Collected data included electrecticity usage at 30 minutes interval, appliances used in the home and different socio-economic information about the people living in a particular house. They clustered each days usage which they call load profiles. A customer's daily usage then can be assigned to one of those load profiles. The customer is then charactersized by the the mostly used load profile. The authors then trained a linear regression classifier that was built upon the socio-economic information of a the customers, types of appliances used in the house and the description of the house to figure out the common load profile of the given household. The predicted load profile of the customer received from the linear regression model will be used to predict the demand of the customer for a given day.\cite{cho1995customer} noticed difference of behavior among customers. They manually clustered the population in four categories namely commercial, office, residendial and industrial customers. In their paper \cite{wang2015gongbroker}, the authors proposed a novel demand prediction mechanism. In powertac competition, every broker is provided with past two weeks usage of all the customers or bootstrap usage. Their proposed broker clustered the customers based on the bootstrap data. For each cluster, the broker would make a linear regression model. The input variables included past average usage and weather related information. This approach of prediction clusters is based on the usage pattern of the customers. So this method may not be suitable for customers with irregular usage pattern such as customers with load shifting capabilities and electric vehicle customers.

\section{engineerign methods to lead forecasting}
%application of kalman filter
The authors \cite{al2004short} have used Kalman Filter to forecast short term load demand. Kalman Filters are used widely to approximate current state of a dynamic system. To do this, it computes the next state of the system using the provided algorithm. Also, it observes what the measurements say about the current state of the system. Both of the prediction mechanisms of the current state has high uncertainity. When they are combined toghther, the uncertainity gets reduced. 

\section{expert system based load forecasting}
%expert systems for load forecasting
The authors in ref \cite{rahman1988expert} have proposed an expert system based load forecasting method for the region Virgina. The expert system would forecast load of upcoming 24 hours. They observed the variables that are likely to affect the load. They came up with variables such as temperature, load of previous hour, season and day of week have strong correlation with the observed load. They implemented a computer program that mimicked how a human operator makes load forecast based on the independent variables. For a specific region's weather condition, their method worked well and required limited amount of historical data.


From the review of the literature, the importance of weather related variables such as temperature, cloud cover and windspeed is evident. Also, hour of the day and day of week are highly correlated with the load demand. Combination of machine learning classifiers and clustering algorithms appears to be a better idea. For the methodlogy of \cite{parra2013initial} it will take a large number of predictors for the simulation system. Also, those predictors will not work if the name of the customer is changed or a new customer is introduced as each predictor is hardcoded with a specific customer. It sounds reasonable to cluster the data first and then train machine learning classifier for each cluster. This apporach will hold generality. Instead of training only on bootstrap data as the \cite{wang2015gongbroker} have done, wealth of data generated from the simulations can be used to  train the cluster. Since the clustering is done offline, this apporach wll not  suffer from the problem of having a time limit that the broker has to face if the cluster is trained during the competition. After the clustering is done, for each cluster, different machine learning classifiers can trained to figure out which one performs the best. So, the broker will no longer sticked to linear regression. This way, the training module will be able to deal with new customers.


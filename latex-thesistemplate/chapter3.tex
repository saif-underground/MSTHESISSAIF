% chap3.tex (Definitions and Theorem)

\chapter{Related Works}



In this chapter I will describe what other works has so far been done to predict customer's future energy demand.

%motivation for good prediction
Predicting customer's energy demand is important becasue failure to predict the demand accurately can cause monetary and environmental loss. Customer's with shifting load can use the energy produced by the renewable energy. Those customers can shift their load to a time where there is a high probablity of renewable energy production.

%Why are we using the powerTAC
Acting directly on the real environment can be risky. The powerTAC simulation system gives a low risk platform where the researcher's can build and test their works before deploying to real world. 

%types of load forecasting
There are different types of load forecasting namely short term load forecasting and long term load forecasting. Short term load forecasting deals with forecasting the customer's demand has the range of time couple of weeks. Long term load forecasting may forecast customer's demand over month or year \cite{cho1995customer}.

%TAC tex 13
TACTEX'13 won the PowerTAC competition in 2013. In a week a customer have 24 * 7 = 168 slots. TACTEX'13 the winner of PowerTAC competition 2013 kept track of average usage of 168 weekly slot for each customer. To predict a future time slot, their agent would look at at which weekly slot the future time slot would fall in. Then the agen uses that weekly slot's average usage as the prediciton of the future slot.

%talking about use of arima model
The ARIMA model uses both moving average and auto regression to forecast the demand. To make a forecast about a future time slot, the auto regression model uses some previously observed time slots values based on its degree. Moving average scheme would use the average of all the known time series data points to make a prediciton about a future time slot .Problem with univariate ARIMA model is that they don't take into account other variables that my affect the demand such as temperature. \cite{cho1995customer} attempted to forecast the energy demand for a region of Taiwan. They found temperature has effect on the energy usage of customers. To make prediction about the demand they used a transfer function that relates the daily temperature with energy usage along with the ARIMA model. This scheme resulted better than the univariate ARIMA model. They manually clustered the population in four categories such as commercial, office, residendial and industrial customers. 

%talk about influence of temperature
Intuitively weather variables such as temperature seem to have some effect on how much energy people use. Researchers have found that weather effect relies on the time duration the training data has. \cite{chen2004load} trained a SVM energy demand predictor that would predict energy demand of customers for the month January. The training data consited of every half hour's electricity demand from 1997 to 1998, average temperature from 1995 to 1998. They trained the predictor with only the portion of data that are related to the month January. They have found that within the month of January the temperature does not vary much and excluding the temperature from the feature set actually gives better prediction. Again, if energy demand is long term which means the window of prediciton is about a year, the temperature seems to have effect on the energy demand of the customers. \cite{hart2004weather} collected data of 18 months from households of a region of Australia. They collected the weather data from weather office and from self transplanted devices. They observed how the household customers use appliances based on the temperature. They came into conclusion that for that region, equilibrium point for energy is usage at temperature 0.25 degree celcius. If the temperature increases or decreases from this temperature, the electricity usage increases. They explained the behavior by stating as the temperature decreases, houses customers tend to use heaters and if the temperature rises they tend to use coolers more.

%talk about regional load forecasting
Regional load forecasting will enable us to know which regions need more energy. If we know which regions need more energy, we will know most suitable places to place electricity generator plants. \cite{hsu2003regional} worked on load forecasting based on region. They diivided electricity usage of Taiwan in 4 areas. For each region, they collected GDP,  population, highest temperature and aggregated load. After that, they trained Artificial Neural Network model for each region. For baseline, they trained linear regression model for each region. The result showed that, the ANN based load forecasting methods performed better than the linear regression  methods. 

%talk about clustering for load forecasting
\cite{mcloughlin2015clustering} have used clustering method to forecast customer's future electricity demand. They collected data from more than 4000 household customers in Ireland for about 6 months. Collected data included electrecticity usage at 30 minutes interval, appliances used in the home and different socio-economic information about the people living in a particular house. They clustered each days usage which they call load profiles. A customer's daily usage then can be assigned to one of those load profiles. The customer is then charactersized by the the mostly used load profile. The authors then train a linear regression classifier that was built upon the socio-economic information of a the customers, types of appliances used in the house and the description of the house to figure out the common load profile of the given household. The predicted load profile of the customer received from the linear regression model will be used to predict the demand of the customer for a given day.
 

%talk about influence of behavior for modeling customers
\cite{lampropoulos2010methodology} the authors proposed a methodology to model electric car user cusotmer's demand. They uses data of electric car users of Netherlands from 2000 to 2007. The data included purpose and starting time of each drive, duration of the drive and the time the driver spent at the purpose destination and when the time when the driver returned home. From this data, the authors found usually the drivers would recharge their car at around hours 17:00 to 19:00. So the electricity generators may find this time a peak demand time due to the added energy demand of the electric vehicle customers. 

%talk about choosing similar days to forecast demand
\cite{liu2006accurate} the authors observed load of certain hour of a certain day is highly correlated with load of some certain days before that day. On the basis of the observation, they would take ten most similar looking days electricity usage and feed it to an ANN to make forecast of the day. The authors found the other variables that may affect the load such as temperature, humidity may change so swiftly that inclusion of them may reduce the accuracy of the predictor. So they excluded all the social, environmental variables from their model of prediction.
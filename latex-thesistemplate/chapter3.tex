% chap3.tex (Definitions and Theorem)

\chapter{Related Works}

In this chapter, I have described different methods of energy load forecasting for long term and short term in the literature. It is hard to know the state of the art electricity consumption mechanism in Power TAC as most of the researchers did not publish their demand forecasting mechanis \cite{liefers2014successful}, \cite{ozdemir2015winner}, \cite{serranofixing}, \cite{ozdemir2015agentude}. So I mostly describe the works done for real world electricity demand forecasting mechanisms.   

\section{Variables in Electricity Demand}

Studies such as \cite{hor2005analyzing}, \cite{hart2004weather} and \cite{cho1995customer} have found that temperature has effect on electricity demand. The study in \cite{hart2004weather} was done in a region of Australia.  It was found that,  in a lower temperature the customers tend to use heaters and in a higher temperature they tend to use coolers. As a result, the increase or decrease of temperature from a certain point will cause the consumption of electricity to increase. In study \cite{cho1995customer}, two demand forecasting models were proposed. One was univariate Auto Regressive Integrated Moving Average (ARIMA) and another one was univariate ARIMA model along with temperature depended transfer function. The model with temperature variable did better forecasting than the one without the temperature variable. On the other hand, the study in  \cite{chen2004load} showed that inclusion of temperature variable in forecasting model actually introduced more error in demand prediction. The aim of the study was to make forecasting about electricity usage of January based on past five years training data using a Support Vector Machine (SVM)  forecaster. The reason behind of getting more error after including temperature variable may be because during Januray the temperature did not change much and the inclusion might have caused overfitting. 

Weather variables such as  wind speed and cloud cover has effect on electricity demand \cite{hor2005analyzing}, \cite{rudenauer2004energy}. As cloud cover increases, the demand for light increases too. The increased lighting demand causes increased electricity demand. The period where cloud cover was low, the electricity demand was also low \cite{hor2005analyzing}. High speed wind across wet walls help cool houses. High speed wind thus may cause reduced electricity demand due to reduced demand of air cooling \cite{rudenauer2004energy}. 


In the survey article \cite{hahn2009electric}, the authors reported that the day of the week and the month of the year is highly correlated with customer's energy demand. The electricity load demand can be higher or lower based on the day of week. The weekends usually have different load demand pattern than the week days. Also, based on the hour of a given day, the load demand can be higher or lower too. The season also showed impact on electricity demand. 

\section{Electricity Load Forecasting Using Statistical Method}

To make electricity load forecast, researchers have used statistical methods such as statistical average, Auto Regressive Integrated Moving Average (ARIMA) and exponential smoothing.  Agent TACTEX'13 \cite{urieli2014tactex}, the winner of the PowerTAC competition in 2013, used the statistical average to make electricity demand forecasting for an hour of a day of a week. In a week a customer has 24 * 7 = 168 hours or slots where it can consume electricity. TACTEX'13  kept track of average usage of 168 weekly slots for each customer. To predict a future time slot, their agent would look at which weekly slot the future time slot would fall in. Then the agent used that weekly slot's average usage as the forecast of the future time slot. \cite{cho1995customer} have used an ARIMA model for load forecasting. The ARIMA model uses both moving average and auto regression to forecast the demand. To make a forecast about a future time slot, the auto regression model uses some previously observed time slots values based on its degree. Moving average scheme would use the average of all the known time series data points to make a prediction about a future time slot. In the study \cite{amjady2001short}, a short term load demand was proposed that uses several ARIMA models. For the combination of week day and temperature level, 16 short term load forecasting models were used. This scheme made better forecasting than a single ARIMA model. The authors in \cite{jalil2013electricity} used modified Halt Winter Exponential Smoothing for demand forecasting. The modified exponential method was cabaple of dealing with weekly and daily seasonality pattern present in the data.    


\section{Load Forecasting using Machine Learning}

Support Vector Machine(SVM) proved to be an effective tool for load forecasting \cite{sapankevych2009time}, \cite{chen2004load}. In \cite{chen2004load}, the authors used SVM to forecast electricity demand of January based on past 5 years electricity consumption data. Separate SVM models for separate were proposed. SVM model trained with data from January was able to make better load forecasting. Artificial Neural Network (ANN) is another favorite load forecasting mechanism among the researchers \cite{izgi2012short}, \cite{quan2014short}, \cite{hsu2003regional}. The author \cite{quan2014short}, used ANN to model Prediction Interval (PI) to forecast renewable energy forecasting.  The author \cite{izgi2012short} used ANN to figure out the time horizon suitable for solar energy production. \cite{parra2013initial} the authors used various machine learning techniques to make 24 hours ahead load forecast for the Power TAC simulation. They found that hour of week, weather related features such as temperature cloud cover were influential to the electricity load demand. The forecasting modules made low error while forecasting for the customers that showed regularity in their energy consumption behavior. The application of linear regression \cite{mcloughlin2015clustering} {hernandez2012classification} and Kalman Filtering \cite{al2004short} also appeared for demand forecasting in the literature. 

\section {Load Forecasting using Clustering}
%talk about clustering for load forecasting 

Clustering can be used to group consumers with same electricity demand pattern \cite{hernandez2012classification}. The authors \cite{mcloughlin2015clustering}, applied clustering on 6 months electricity usage of household consumers in Ireland. Application of clustering generated common load patterns called load profiles which were used to forecast about future load demand. The authors \cite{cho1995customer} noticed that customers can be categorized to improve accuracy of demand forecasting. They manually clustered the customers in four groups namely, commercial, office, residential and industrial customers. For Power TAC environment, the authors \cite{wang2015gongbroker} proposed a broker that clusters the bootstrap data of customers and generates a linear regression classifier for demand prediciton for each cluster.  The authors \cite{hernandez2012classification} used kMeans and Self Organaizing Map to cluster industrial park's consumption in Spain to understand micro environments present in a larger environment.


\section{Expert System based Load Forecasting}
%expert systems for load forecasting

Expert system based electricity demand prediction contains variables that are likely to affect electricity demand \cite{rahman1988expert}, \cite{ho1990short}. This system then mimicks a human operator's steps to load forecast. This load forecasting mechanism appeared applicable for short term load forecasting \cite{rahman1988expert}, \cite{ho1990short}, \cite{moghram1989analysis}. \\

From the review of the literature, the importance of weather related variables such as temperature, cloud cover and wind speed is evident. Also, the hour of the day and day of the week are highly correlated with the load demand. A combination of machine learning classifiers and clustering algorithms appears to be a better idea. For the methodology of \cite{parra2013initial} it will take a large number of predictors for the simulation system. Also, those predictors will not work if the name of the customer is changed or a new customer is introduced as each predictor is hard coded with a specific customer. It sounds reasonable to cluster the data first and then train machine learning classifier for each cluster. This approach will hold generality. Instead of training only on bootstrap data as proposed in \cite{wang2015gongbroker}, a wealth of data generated from the simulations can be used to  train the cluster. Since the clustering is done offline, the proposed approach will not  suffer from the problem of having a time limit that the broker has to face if the cluster is trained during the competition. After the clustering is done, for each cluster, different machine learning classifiers can be trained to figure out which one performs the best. So, the broker will no longer stick to linear regression as in \cite{wang2015gongbroker}.


% chap3.tex (Definitions and Theorem)

\chapter{Related Works}



In this chapter I will describe what other works has so far been done to predict customer's future energy demand.

Predicting customer's energy demand is important becasue failure to predict the demand accurately can cause monetary and environmental loss. 

For a single customer's energy demand prediction, some work  have been done. Several machine learning algorithms have been applied to predict the future demand and to figure out which attributes have the most predictive power. In thid methodology if we are to make prediciton about n customers, we will have to have n different type of models which may be infeasible for a large value of n. But they don't tell what happens if there are different types of customers instead of a single type customer. Some works on demand prediction based on the aggreagated population of an area has been done. In a wide area my types of customers may reside and their power demand may vary widely. This method fails to fine grain the customer according to their behavior. Some researchers have used prediction as a mean to cluster customer based on their energy usage behavior. The uses a long history of the customer's energy usage and try to cluster customers based on the huge amount of dimensions. As we know using a lot of dimensions or attributes in machine learning may lead to to curse of dimensionality problem. Which is undesirable.


%types of load forecasting
There are different types of load forecasting namely short term load forecasting and long term load forecasting. Short term load forecasting deals with forecasting the customer's demand has the range of time couple of weeks. Long term load forecasting may forecast customer's demand over month or year \cite{cho1995customer}.

%TAC tex 13
TACTEX'13 won the PowerTAC competition in 2013. In a week a customer have 24 * 7 = 168 slots. TACTEX'13 the winner of PowerTAC competition 2013 kept track of average usage of 168 weekly slot for each customer. To predict a future time slot, their agent would look at at which weekly slot the future time slot would fall in. Then the agen uses that weekly slot's average usage as the prediciton of the future slot.

%talking about use of arima model
The ARIMA model uses both moving average and auto regression to forecast the demand. To make a forecast about a future time slot, the auto regression model uses some previously observed time slots values based on its degree. Moving average scheme would use the average of all the known time series data points to make a prediciton about a future time slot .Problem with univariate ARIMA model is that they don't take into account other variables that my affect the demand such as temperature. \cite{cho1995customer} attempted to forecast the energy demand for a region of Taiwan. They found temperature has effect on the energy usage of customers. To make prediction about the demand they used a transfer function that relates the daily temperature with energy usage along with the ARIMA model. This scheme resulted better than the univariate ARIMA model. They manually clustered the population in four categories such as commercial, office, residendial and industrial customers. 

%talk about influence of temperature
Intuitively weather variables such as temperature seem to have some effect on how much energy people use. Researchers have found that weather effect relies on the time duration the training data has. \cite{chen2004load} trained a SVM energy demand predictor that would predict energy demand of customers for the month January. The training data consited of every half hour's electricity demand from 1997 to 1998, average temperature from 1995 to 1998. They trained the predictor with only the portion of data that are related to the month January. They have found that within the month of January the temperature does not vary much and excluding the temperature from the feature set actually gives better prediction. Again, if energy demand is long term which means the window of prediciton is about a year, the temperature seems to have effect on the energy demand of the customers. \cite{hart2004weather} collected data of 18 months from households of a region of Australia. They collected the weather data from weather office and from self transplanted devices. They observed how the household customers use appliances based on the temperature. They came into conclusion that for that region, equilibrium point for energy is usage at temperature 0.25 degree celcius. If the temperature increases or decreases from this temperature, the electricity usage increases. They explained the behavior by stating as the temperature decreases, houses customers tend to use heaters and if the temperature rises they tend to use coolers more.

 //


% talk about some motivation
Customer's with shifting load can use the energy produced by the renewable energy. Those customers can shift their load to a time where there is a high probablity of renewable energy production. 

%Why are we using the powerTAC
Acting directly on the real environment can be risky. The powerTAC simulation system gives a low risk platform where the researcher's can build and test their works before deploying to real world. 